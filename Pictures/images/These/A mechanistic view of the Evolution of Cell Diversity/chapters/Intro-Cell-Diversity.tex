\textit{“It is not the strongest of the species that survives,
not the most intelligent that survives.
It is the one that is the most adaptable to change.”}

Charles Darwin



‘‘\textit{The limits of my language are the limits of my world}''.

\citep{Wittgenstein22}

Evolution is the ongoing process that gave birth to the wide zoo of organisms that have ever lived and shall ever live \citep{Darwin59,Wallace58}. It is, as it were, the language of Nature when it deals with mutable self-replicating entities. If this language has been fruitful, it still sets the limits of what can exist and how this space of possibilities is explored, as pinpointed by \citet{Wittgenstein22}. Melting many mechanisms acting at different levels of space and time, these limits are yet to be fully understood. It has long been known indeed that the power of Natural Selection is limited by genetic drift \citep{Wright30} such that organisms are the best of the possible ones under the conflicting mutational and selective pressures \citep{Kimura62,Ohta92}, but how Evolution combines with the complex genetics underlying traits remains, to say the least, inchoate \citep{Hansen13,Blanquart16,Barton17,Harpak21}.

\textit{“If one could conclude as to the nature of the Creator from a study of creation it would appear that God has an inordinate fondness for stars and beetles.”}

John Burdon Sanderson Haldane

\textit{“An attempt to study the evolution of living organisms without reference to cytology would be as futile as an account of stellar evolution which ignored spectroscopy.”}

John Burdon Sanderson Haldane

\textit{“But if thought corrupts language, language can also corrupt thought.”}

George Orwell

\textit{"The Darwinian process of continued interplay of a random and a selective process is not intermediate between pure chance and pure determinism, but qualitatively utterly different from either in its consequences."}
Sewall Wright