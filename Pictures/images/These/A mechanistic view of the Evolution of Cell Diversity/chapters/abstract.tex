\Large
 \begin{center}
Résumé \\ 
\hspace{20pt}

\end{center}

\normalsize

Abstract en français

\newpage

\Large
 \begin{center}
Abstract \\ 
\hspace{20pt}

\end{center}

\normalsize

Since Life was born, its Evolution has created an exceptional diversity of entities spanning an extravagant range of sizes from tiny microscopic molecules to the giant organisms that embody Megafauna. This broad variability, which exists both between and within classes of biological entities (eg. proteins), has often been theoretically explained  by  assuming the existence of biological trade-offs – impossibility to optimise many traits at once - and/or specific niches (eg. two different nutrients in the environment). However, how these trade-offs build up at the cellular level has mostly remained elusive because models  of specialisation overlook the very mechanistic underpinnings of cells, that is to say how they actually work. Here, we develop a model where the fitness of cells emerges from a sequence of enzyme-substrate reactions that each produce a specific metabolite like ATP, and first show that accounting for physical, ecological and cellular constraints sheds light on the reasons why enzyme properties  resemble a zoo although they seemingly evolve under a similar directional selective pressure – and should thus, at first glance, all look the same. Based on these metabolic fitness landscapes and Adaptive Dynamics, we then simulate cell competition to demonstrate how the simple and intrinsic physical constraint of membrane permeability can explain the emergence of cell diversification - further followed by cell specialisation – even within constant environments, a poorly understood phenomenon known as cross-feeding. Such a model enables to predict which intermediate metabolites should give rise to cross-feeding and we highlight that the available data strikingly match our predictions. Finally, we conclude by discussing about the evolutionary accessibility of these phenotypes  and sketch a final step of this project that attempts to determine whether this cell specialisation is more likely to occur through non-genetic diversified bet-hedging or genetic polymorphism/speciation.

\newpage